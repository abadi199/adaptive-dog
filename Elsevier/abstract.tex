%!TEX root = main.tex
	Non-photorealistic rendering is a method of imitating hand-drawn images using computer as the tool. One of hand-drawn techniques used by artists is line drawing. This goal of this thesis is to produce a technique to create line drawing images from photographs. Based on a previous technique called Flow-based Difference of Gaussian (FDoG), we try to improve the output image so that it will create more believable pictures, as it if were hand-drawn by an artist.

FDoG has proven to be able produce images with coherent and continues lines, but it fails to capture coarse details on isotropic areas in the image. Another technique in line drawing called Difference of Gaussian (DoG), which FDoG was based on, can produce better detail on isotropic areas. Combining these two techniques can create better results for both isotropic and anisotropic areas. We create an image segmentation technique using polarity to divide isotropic and anisotropic areas in the image. Using this segment, we then adaptively apply FDoG and DoG filters to each segment.