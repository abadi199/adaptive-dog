\chapter{Introduction}

Several quantum mechanical phenomena can be described in terms of simple
harmonic oscillators modified by the inclusion of linear friction.  For
example, deep inelastic scattering and field-theoretic formulations of
fundamental anyons can be so modelled.  It has recently been shown \cite{emmg}
that squeezed light entering quantum optics can be statewise identified up to
automorphisms with damped simple harmonic oscillators.  An understanding of 
such oscillators is therefore of use.

The frictionally damped simple harmonic oscillator can be expressed in a
Lagrangian formulation only with the addition of a second variable of motion.
A suitable Lagrangian is known to be \cite{mf:mtp} 

\begin{equation}
L = m \dot{x} \dot{y} + \frac{R}{2} (x \dot{y} - \dot{x} y) - k x y
\end{equation}

where $R$ characterizes the strength of the friction.  Canonical momenta and
a Hamiltonian can be obtained in the usual manner and canonical quantization
carried out.  Feshbach and Tikochinsky have shown \cite{ft:fest} that the 
eigenvalues of the resulting Hamiltonian are

\begin{equation}
H = \hbar \Omega (n_{A} - n_{B}) \pm \frac{i \hbar R}{2 m} (n_{A} + n_{B} + 1)
\end{equation}

where $\Omega^{2} \equiv \frac{k}{m} - \frac{R^{2}}{4 m^{2}}$ and $n_{A}$ and
$n_{B}$ range over the nonnegative integers.  The complexity of the eigenvalues
is expected, being characteristic of nonconservative systems, and causes no
difficulties in nonstatistical questions.

Problems arise when an attempt is made to model the behavior of an ensemble
of these oscillators.  A first attempt follows the normal procedure, used for
conservative systems (in which the Hamiltonian is equivalent to the energy),
to calculate the partition function for a canonical formulation as

\begin{equation}
Z = \Sigma e^{- \beta H}
\end{equation}

where $H$ ranges over the Hamiltonian eigenvalues; such quantities as
temperature, internal energy, and entropy are determined from the form of $Z$.

Two difficulties are immediately apparent.  The first is that the real part of
the Hamiltonian eigenvalues has no lower bound, as $n_{A}$ can be held fixed
while $n_{B}$ ranges indefinitely upwards, and so the partition function sum
diverges.  The second is that even if some cutoff is made, $Z$ is no longer
positive definite, since $e^{- \beta H}$ can be negative for complex $H$
(its reality is assured, since the sum includes the complex conjugate of
each of its summands).

The problem is, simply, to find some suitable method of applying canonical 
statistics to such ensembles, and then to reach an understanding of any unusual
features in the thus-calculated quantities --- for instance, it appears
possible that complexity will enter such things as the entropy, possibly
indicating some sort of time evolution.  

I shall give a set of criteria for a solution.  I shall then go over several
of the approaches we took, giving motivations and results of
each.  Finally, I shall present a solution in which the reasons for using
Hamiltonian eigenvalues at all are examined and found wanting, motivating the
use of energy eigenvalues instead.

\newpage
\chapter{Criteria}

An acceptable solution must yield approximately as much information about
ensembles of the damped oscillators as normal methods do about ensembles of
undamped oscillators.  It must allow us to calculate thermodynamic quantities
such as temperature, entropy, and energy.  However, it is acceptable that
it do so only with the aid of approximations.  

In the $R = 0$ (no damping) case, a solution must yield the same results as
those already known for the undamped situation.  Further, we expect that the
results for small $R$ be close to these.  More generally, the derived
quantities should be continuous functions of $R$.

The latter criterion is only a fragment of a larger one, namely that the
derived quantities must make sense.  All differences between the damped and
undamped cases must be explained.  It is especially important that there be
plausible physical and mathematical interpretations of any oddities that may
appear in such quantities, such as complexity.  

Finally, the derivation of the solution must itself be plausibly justifiable
on both physical and mathematical grounds.

\newpage
\chapter{Attempts}

\section{Separating the Hamiltonian}

\subsection{Motivation}

We observe that the eigenvalues of the Hamiltonian can be rewritten as

\begin{equation}
H = (\hbar \Omega \pm \frac{i \hbar R}{2 m}) n_{A} -
    (\hbar \Omega \mp \frac{i \hbar R}{2 m}) n_{B}.
\end{equation}

Since this is linear in $n_{A}$ and $n_{B}$, it suggests that the Hamiltonian
may also separate, as $H = H_{A} - H_{B}$.  An appealing symmetry leads us
to define

\begin{equation}
\alpha \equiv \Omega \pm \frac{i R}{2 m} \equiv \omega e^{i \theta}
\end{equation}

where

\begin{equation}
\theta \equiv \pm \arctan \frac{R}{2 m \Omega}
\end{equation}

so that we may write the decoupled eigenvalues as

\begin{equation}
H_{A} = \hbar \alpha (n_{A} + \frac{1}{2}), \hspace{1cm}
	H_{B} = \hbar \alpha^{*} (n_{B} + \frac{1}{2})
\end{equation}

for each sign of $\theta$,
so that each looks like the spectrum for an undamped SHO, rotated in the
complex plane, and the zero-point energies have appeared.  Interpreting the
Hamiltonian eigenvalues as the difference of two more normal spectra gives
a reason for the lack of a zero-point value; there are two, which cancel.

While this does not logically imply that the Hamiltonian {\em operator} will
separate into two parts, it makes it plausible.  We therefore seek a 
transformation that will yield a Hamiltonian with no explicit coupling term.
If we find such a Hamiltonian, we can be sure that we can treat the system
as two separate systems.

\subsection{Attempts}

The obvious transformation is
$u \equiv \frac{x + y}{\sqrt{2}}, v \equiv \frac{x - y}{\sqrt{2}}$
which suggests itself as a way to decouple the $p_{x}p_{y}$ and $xy$ terms.
This yields 

\begin{equation}
L = \frac{m}{2} (\dot{u}^2 - \dot{v}^2) + \frac{R}{2} (\dot{u} v - u \dot{v})
	- \frac{k}{2} (u^2 - v^2),
\end{equation}
\begin{equation}
H = \frac{p_{u}^2}{2 m} - \frac{p_{v}^2}{2 m} + 
	\frac{R}{2 m}(u p_{v} - v p_{u}) + \frac{m \Omega^2 u^2}{2} - 
	\frac{m \Omega^2 v^2}{2}
\end{equation}

which are decoupled for the $R = 0$ case only.  In this case the system 
presents itself as the difference of two SHOs; letting $v \Rightarrow i v$, 
it becomes a sum.
Although it was not expected to achieve separation, we then looked at the
further ``angular'' transform $u = \rho cos \theta, v = \rho sin \theta$.  
This yields

\begin{equation}
L = \frac{m}{2} \dot{\rho}^2 + \frac{m \dot{\theta}^2 - i R \dot{\theta}
	- k}{2} \rho^2,
\end{equation}
\begin{equation}
H = \frac{3 p_{\theta}^2}{2 m \rho^2} + \frac{p_{\rho}^2}{2 m} 
	+ \frac{i R p_{\theta}}{2 m} - \frac{m \Omega^2 \rho^2}{2}
\end{equation}

which bear the expected resemblance to the normal angular form of the
two-dimensional SHO.  While this form allows the separation $\psi(\rho, \theta)
= F(\rho) e^{i \lambda \theta}$, the Hamiltonian does not separate into 
$H_{\rho} + H_{\theta}$.

It should be noted that any quantization undertaken after letting 
$v \Rightarrow i v$ is suspect, as the change in reality of variables 
necessarily corresponds to a change in which variables are observable and
which nonphysical.  In particular, the calculation of the eigenvalues of
the number operators as the whole numbers depends on the creation and
destruction operators being adjoint, so that $<\psi| N |\psi>$ has a positivity
requirement; the loss of the reality of $x$ and $y$ destroys this.  In fact,
if quantization is carried out on the $uv$ system after $v \Rightarrow i v$
(meaning that $u$ and $v$ are real, since they are being quantized, and that
therefore $x$ and $y$ are complex and conjugate to each other), the 
eigenvalues of the number operator taken from $x$ and $y$ become the 
nonpositive integers.  The eigenstates of the Hamiltonian therefore change;
what previously formed the eigenstates are now unphysical.  This may or may
not be damaging, but it shows that care must be taken with such operations.

\newcommand{\qp}{\mbox{$q_{+}$}}
\newcommand{\qm}{\mbox{$q_{-}$}}
\newcommand{\uv}{\mbox{$q_{\pm}$}}
\newcommand{\vu}{\mbox{$q_{\mp}$}}
\newcommand{\xu}{\mbox{$\frac{\partial x}{\partial \qp}$}}
\newcommand{\xv}{\mbox{$\frac{\partial x}{\partial \qm}$}}
\newcommand{\xuv}{\mbox{$\frac{\partial x}{\partial \uv}$}}
\newcommand{\xvu}{\mbox{$\frac{\partial x}{\partial \vu}$}}
\newcommand{\ddx}{\mbox{$\frac{\partial^2 x}{\partial \qp \partial \qm}$}}
\newcommand{\ddy}{\mbox{$\frac{\partial^2 y}{\partial \qp \partial \qm}$}}
\newcommand{\yu}{\mbox{$\frac{\partial y}{\partial \qp}$}}
\newcommand{\yv}{\mbox{$\frac{\partial y}{\partial \qm}$}}
\newcommand{\yuv}{\mbox{$\frac{\partial y}{\partial \uv}$}}
\newcommand{\yvu}{\mbox{$\frac{\partial y}{\partial \vu}$}}
\newcommand{\ux}{\mbox{$\frac{\partial \qp}{\partial x}$}}
\newcommand{\uy}{\mbox{$\frac{\partial \qp}{\partial y}$}}
\newcommand{\vx}{\mbox{$\frac{\partial \qm}{\partial x}$}}
\newcommand{\vy}{\mbox{$\frac{\partial \qm}{\partial y}$}}
\newcommand{\uvx}{\mbox{$\frac{\partial \uv}{\partial x}$}}
\newcommand{\uvy}{\mbox{$\frac{\partial \uv}{\partial y}$}}
\newcommand{\ud}{\mbox{$\dot{\qp}$}}
\newcommand{\vd}{\mbox{$\dot{\qm}$}}
\newcommand{\uvd}{\mbox{$\dot{\uv}$}}
\newcommand{\pu}{\mbox{$p_{+}$}}
\newcommand{\pv}{\mbox{$p_{-}$}}
\newcommand{\puv}{\mbox{$p_{\pm}$}}

The form of the Hamiltonian eigenvalue spectrum leaves us still optimistic
that separating the Hamiltonian into $H_{A} - H_{B}$
is possible.  However, we discovered that
no point transformation can accomplish this.  Consider an arbitrary 
reversible point transformation $\qp = \qp(x, y), \qm = \qm(x, y),
x = x(\qp, \qm), y = y(\qp, \qm).$  The Lagrangian becomes

\begin{equation}
L = m (\xu \ud + \xv \vd)(\yu \ud + \yv \vd) + \frac{R}{2} (x (\yu \ud +
	\yv \vd) - y (\xu \ud + \xv \vd)) - k x y
\end{equation}

from which we get

\begin{equation}
\puv \equiv \frac{\partial L}{\partial \uv} = 2 m \xuv \yuv \uvd +
	\frac{R}{2} (x \yuv - y \xuv)
\end{equation}

and 

\begin{equation}
H = \frac{p^{2}_{+}}{4 m} \ux \uy + \frac{R}{4 m} (y \uy - x \ux) \pu
	+ \frac{p^{2}_{-}}{4 m} \vx \vy + \frac{R}{4 m} (y \vy - x \vx) \pv
	+ m \Omega^2 x y.
\end{equation}

Necessary and sufficient conditions for $H$ to separate into $H_{+} + H_{-}$
are therefore that $\uvx \uvy$ and $y \uvy - x \uvx$ are functions of
$\uv$ only (keeping the signs consistent), and that $xy$ be the sum of a 
function of $\qp$ only and a function of $\qm$ only.  In exact form, these
conditions become

\begin{equation}
\yuv \ddx + \xuv \ddy = 0, \label{eq:ab}
\end{equation}
\begin{equation}
(x \ddx - \xu \xv) (\yuv)^2 - (y \ddy - \yu \yv) (\xuv)^2 = 0,\label{eq:cd}
\end{equation}
\begin{equation}
y \ddx + x \ddy = 0.
\end{equation}

From eq.s~\ref{eq:ab} we see that since 
$\ddx = - \frac{\xuv}{\yuv} \ddy$ for both sign choices, either 
$\frac{\xu}{\yu} = \frac{\xv}{\yv}$ or $\ddy = \ddx = 0$.  In the latter
case eq.s~\ref{eq:cd} become

\begin{equation}
\yuv \yvu (\xuv)^2 - \xuv \xvu (\yuv)^2 = 0
\end{equation}

which we can divide through by $\yuv \xuv$ to show that

\begin{equation}
\xu \yv = \xv \yu.
\end{equation}

Therefore we see that separation of the Hamiltonian requires that
$\xu \yv = \xv \yu$.  Since $\xu \yv + \xv \yu$ is identically zero,
this requirement means that $\xu \yv = \xv \yu = 0$.  This, in turn, means 
that either
$\xu = \xv = 0$, in which case $x$ is trivial; $\yv = \yu = 0$, in which
case $y$ is trivial; or $\xuv = \yuv = 0$ for one sign, in which case 
$\uv$ drops out of the equations for $x(\qp, \qm), y(\qp, \qm)$.  
Since any of these would make the transformation $x, y
\Rightarrow \qp, \qm$
irreversible, we see that no point transform will achieve separation of the
Hamiltonian.

\section{Recoupling the Hamiltonian}

\subsection{Motivation}

As noted above, the form of the Hamiltonian spectrum suggests that it is
composed of two coupled parts.  If we take this as given, even without
explicitly separating the Hamiltonian operator, we may suppose that the
minus sign between the parts is indicative only of the form of the coupling
and is not important to the statistics of the system.  If so, we can recouple
the Hamiltonian as $H = H_{A} + H_{B}$, yielding values

\begin{equation}
H = \hbar \Omega (n_{A} + n_{B} + 1) \pm \frac{i \hbar R}{2 m} (n_{A} - n_{B})
\end{equation}

The problem of the immediate divergence of $Z$ is now solved, since $(n_{A} +
n_{B} + 1)$ {\em does} have a lower limit, and the unbounded $(n_{A} - n_{B})$
is now safely imaginary.

Continuing to use subsystems $A$ and $B$, we see
that $Z$ is the product of $Z_{A}$ and $Z_{B}$.  Further, $Z_{A} = Z_{B}^{*}$
and so $Z = |Z_{A}|^{2}$.  Reflecting that the variable $y$ is fictitious, we
may further suppose that this partition function is too large by a power of
two, and therefore use $Z = |Z_{A}| = |Z_{B}|$ only.  (This change is minor,
being equivalent to halving the number of particles in the ensemble, and so
not changing any dependencies.)  It now appears that the effect of the damping,
statistically, is to rotate the undamped spectrum in the complex plane, use
the rotated values in the partition function, and then rotate back (not by
the same amount) to the real axis.  This has a certain intuitive appeal.

\subsection{Discussion}

The first quantity calculated from $Z$ is the internal energy, $E = -
\frac{d}{d \beta} ln Z$.  We immediately encounter trouble:  for certain
combinations of the parameters, $E < 0$.  Since the real part of the recoupled
Hamiltonian is positive for each particle, this seems odd.  An explanation is
found in the fractional occupation numbers for the states $k$,
$\frac{e^{- \beta H_{k}}}{Z}$.  The problem is immediately apparent; while
we have arranged for $Z$ to be convergent, real, and positive, the numerator
ranges over the complex plane.  For $H_{k}$ with imaginary parts of appropriate
magnitude, the fractional occupation is negative; rarely is it purely real.
Most $k$ will have complex occupation numbers.

The meaning of the imaginary part of a complex occupation number could perhaps
be sought in a Fourier transformation with time.  However, no such option is
readily available for the purely negative occupation numbers.  This problem
was left open-ended; the recoupling idea was found to be useful in several
of our other attempts.

\section{Restricting the Ensemble}

\subsection{Motivation}

Because the $y$ dimension is spurious, it seems reasonable to attempt to
eliminate it before doing statistics.  The possible evolutions in $y$-space
are the time-reverse motions of $x$-space.  In fact, the $y$ variable could
be eliminated from the Lagrangian formulation by relying directly on the
action principle.  Instead of using a monogenic Lagrangian as the integrand,
we write $x$ as $x(t)$ and $y$ as $x(-t)$.  This, however, does not lend
itself to Hamiltonian reformulation.  It may indicate that the $x$ and $y$
variables should not be considered as independent as they seem.  One possible
way to eliminate this dependence, in ensembles, is to require that a particle
can be present in a state only if another particle is present in the
time-reversed state, that is, the state with the conjugate eigenvalue.  As this
is simply a restriction on the initial conditions of the system, it is 
perfectly compatible with the Lagrange variational principle.

\subsection{Results}

With this pairing of particles with their time-reverses requirement, it seems 
reasonable to use the sums of the eigenvalues
of the paired particles in the partition function, since the pairs are now
the fundamental unit.  The imaginary parts now simply vanish.  Up to a
few factors of 2, we are left with exactly the statistics of the undamped
oscillator except that $\omega$ is replaced with $\Omega.$  We are now assured
that, at least in the underdamped case, all thermodynamic variables will be
as accessible and meaningful as in the undamped case.  The overdamped and
critically damped cases require further analysis, as some quantities take on
strange values for the former and diverge for the latter.  Our attention is
focused on the underdamped case.

A slightly different path to this result takes the time-reverse symmetry
as requiring that $n_{A} = n_{B}$.  If this restriction is performed only
after recoupling the Hamiltonian, the eigenvalues become, as above, those
of the undamped oscillator with $\omega \Rightarrow \Omega$ with but a
single unwanted factor of 2.  (The previous restriction is the same as
recoupling the Hamiltonian and then requiring that if $(n_{A}, n_{B})$ is
present so is $(n_{B}, n_{A})$.)

\subsection{Discussion}

Elegant though this solution may be, it still has several problems.
First, no way has been found to mathematically verify the
assertions regarding time-reverse linkage with conjugate eigenvalues.
Second, it is unclear that the proper weights for pairs of particles in the
partition function are the sums of the particles' weights, even should the
pairing be accomplished.  Third, it is only somewhat physically plausible
that the effect of the damping should be, statistically, to make the
oscillators act as if they had smaller energy quanta.  Further investigation
may provide answers to these questions.

During the tail end of the research we encountered the preprint \cite{action}
in which the action for the damped SHO is calculated.  Of interest to us are
the final results of the real and imaginary parts of the action as

\begin{equation}
{\cal R}e {\cal A}[x, y] = \int^{t_{f}}_{t_{i}} dt {\cal L},
\end{equation}
\begin{equation}
{\cal L} = \mu \dot{x} \dot{y} - [V(x + \frac{1}{2} y) - V(x - \frac{1}{2} y)]
	+ \frac{1}{2}[x F^{ret}_{y} + y F^{adv}_{x}],
\end{equation}
\begin{equation}
{\cal I}m {\cal A}[x, y] = \frac{1}{2 \hbar} \int^{t_{f}}_{t_{i}} 
	\int^{t_{f}}_{t_{i}} dt ds N(t - s) y(t) y(s)
\end{equation}

(\cite{action}, eq. 30).  In particular, the authors note that ``The classical
constraint $y = 0$ occurs because nonzero $y$ yields an `unlikely process' in
view of the large imaginary part of the action (in the classical `$\hbar 
\Rightarrow 0$' limit) implicit in [the above equations]'' (\cite{action} p10).
While this restriction does not directly obtain in the quantum level, any
restriction that {\em is} imposed on the quantum ensemble must be, when the 
ensemble is taken to the classical $\hbar \Rightarrow 0$ limit, compatible
with this $y = 0$ restriction.  This should be kept in mind in any further
exploration of this approach.

\section{Studying Time Evolution}

It was thought that knowledge of the exact time evolution of the Heisenberg
operators might be useful in inspiring an interpretation of the system.

As stated in \cite{ft:fest}, the metric of the system must be altered so that
time-reverse replaces adjoint in order to achieve normalizability.  We
therefore have the Schr\"odinger equations

\begin{equation}
i \hbar \frac{d |\psi(t)>}{d t} = H \psi(t), \hspace{1cm} i \hbar \frac{d <\psi^{T}|}{d t} =
	<\psi^{T}| H^{T}
\end{equation}

with which we can derive the Heisenberg equation of motion by examining the
expectation value of an operator with no explicit time dependance in
the usual fashion.

Having thus justified use of the usual Heisenberg equation of motion, we find
the fundamental operators to evolve as

\begin{equation}
\dot{A} = \frac{1}{i \hbar} [A, H] = - i \Omega A + \frac{R}{2 m} B^{\dagger},
\hspace{1cm}
\dot{B} = \frac{1}{i \hbar} [B, H] = i \Omega B + \frac{R}{2 m} A^{\dagger}.
\end{equation}

These coupled equations are easily solved to yield

\begin{equation}
A = (A_{0} cosh\frac{R t}{2 m} + B_{0}^{\dagger} sinh\frac{R t}{2 m}) e^{- i
        \Omega t},
\hspace{1cm}
B = (B_{0} cosh\frac{R t}{2 m} + A_{0}^{\dagger} sinh\frac{R t}{2 m}) e^{i
        \Omega t},
\end{equation}

from which we directly verify that the CCR  $[A, B] = 0$ and
$[A, A^{\dagger}] = [B, B^{\dagger}] = 1$ hold true at all time $t$.
Backtracking to the original creation and destruction operators for
$x, p_{x}$ and $y, p_{y}$ via the definitions $A = \frac{a + b}{\sqrt{2}},
B = \frac{a - b}{\sqrt{2}}$ yields

\begin{equation}
a = (a_{0} cosh\frac{R t}{2 m} + a_{0}^{\dagger} sinh\frac{R t}{2 m}) 
	cos \Omega t - i (b_{0} cosh\frac{R t}{2 m} - b_{0}^{\dagger}
	sinh \frac{R t}{2 m}) sin \Omega t,
\end{equation}
\begin{equation}
b = (b_{0} cosh\frac{R t}{2 m} - b_{0}^{\dagger} sinh\frac{R t}{2 m})
	cos \Omega t - i (a_{0} cosh\frac{R t}{2 m} + a_{0}^{\dagger}
	sinh\frac{R t}{2 m}) sin \Omega t
\end{equation}

which we substitute into $a = \frac{1}{\sqrt{2 \hbar \Omega}} (\frac{p_{x}}
{\sqrt{m}} - i \sqrt{m} \Omega x), \hspace{.25cm} b =  \frac{1}{\sqrt{2 \hbar
\Omega}} (\frac{p_{y}}{\sqrt{m}} - i \sqrt{m} \Omega y)$ to obtain our 
final results

\begin{equation}
x(t) = (x(0) cos\Omega t + \frac{1}{m \Omega} p_{y}(0) sin\Omega t)
        e^{- \frac{R t}{2 m}}
\end{equation}
\begin{equation}
p_{x}(t) = (p_{x}(0) cos\Omega t - m \Omega y(0) sin\Omega t)
        e^{\frac{R t}{2 m}}
\end{equation}
\begin{equation}
y(t) = (y(0) cos\Omega t + \frac{1}{m \Omega} p_{x}(0) sin\Omega t)
        e^{\frac{R t}{2 m}}
\end{equation}
\begin{equation}
p_{y}(t) = (p_{y}(0) cos\Omega t - m \Omega x(0) sin\Omega t)
        e^{- \frac{R t}{2 m}}
\end{equation}

which have reasonable forms given the nature of this system.  This information
proved useful in subsequent calculations.

\section{Studying Correlations}

The correlation functions were defined as

\begin{equation}
corr(\Theta) = \frac{<\psi | \Theta(0) \Theta(t) | \psi> - 
	<\psi | \Theta(0) | \psi><\psi | \Theta(t) | \psi>}
	{<\psi | \Theta^{2}(0) | \psi>}
\end{equation}

where $|\psi>$ is a stationary state,
so that for the normal formalism of the undamped simple harmonic oscillator
we have the usual

\begin{equation}
corr(p) = cos \omega t + \frac{i sin \omega t}{2 n + 1}.
\end{equation}

Using the equations of motion given above, the correlations become

\begin{equation}
corr(p_{x}) = e^{\frac{R t}{2 m}} cos\Omega t,
\hspace{.5cm}
corr(p_{y}) = e^{-\frac{R t}{2 m}} cos\Omega t,
\hspace{1cm}
corr(\vec{p}) = 2 cosh\frac{R t}{2 m} cos\Omega t.
\end{equation}

The only apparent characteristic of interest of these equations is that while
as $R \Rightarrow 0$ they become the real part of the usual undamped
correlation, they continue to lack any imaginary part.  This is somewhat
intriguing, especially since the interpretation of imaginary quantities as
Fourier-transformed time evolution would require the complexity to be present
only in the damped case rather than only in the undamped case, rather than
the reverse.  

It is quite possible that the imaginary part is simply
an artifact of the Lagrangian form and has no physical significance.  It is
more likely to be related to the zero-point energy, especially since in the
classical limit $n \Rightarrow \infty$ it vanishes; if this is the case then
its lack is probably due to the zero-point contributions of the two decoupled
systems cancelling in the difference system.

\newpage
\chapter{Solution:  Discarding the Hamiltonian}

\section{Motivation}

Examination of \cite{ft:fest} reveals a small error in terminology:  the
Hamiltonian is identified with the energy.  While this identification is of
course valid in most (i.e. conservative) systems, it fails for nonconservative
systems such as the damped SHO.  In fact, looking more closely, we realize
that since the Lagrangian was not derived as $T - V$ but simply as an
expression that gives the desired equation of motion, there is no reason to
believe the derived Hamiltonian will have anything to do with the energy.
Indeed, even for the $R = 0$ case the eigenvalue spectra of the Hamiltonian
and the energy differ.  Restricting the states to those annihilated by $B$,
as suggested in \cite{ft:fest} for recovery of the normal case, the Hamiltonian
eigenvalues are $\hbar \omega n$.  These lack the zero-point energy even though
we have eliminated the $B$ system.

This, in turn, leads to an examination of the role of the Hamiltonian
eigenvalues in statistical mechanics.  Textbooks on the subject use the terms
``Hamiltonian'' and ``energy'' interchangeably, since they are identical for
the systems these books are concerned with, leaving doubt as to what aspect
of the Hamiltonian makes it relevant.  That is, is it that it governs the
time evolution, or that it is the energy?

Correct time evolution is given by an infinite number of Hamiltonians. 
(Everything in this paragraph holds true for corresponding Lagrangians.)
To be crude, if $H$ produces correct evolution, so does $H' \equiv c_{1} H + 
c_{2}, c_{1}, c_{2} \in C$.  While the eigenstates of any such Hamiltonian 
are, of course, the same, the eigenvalues differ.  Further, introduction of 
spurious variables can alter the spectrum even more; for instance, use of the
two-variable Hamiltonian removes both the zero-point energy and the lower
bound for the undamped SHO.  If the partition sum is to use the eigenvalue
spectrum of a time-evolution operator, it must be possible to uniquely specify
which such operator.  Otherwise, two identical systems in equilibrium with
each other could be mathematically described with different time-evolution
operators, yielding different partition sums, yielding different values of
thermodynamic quantities, contradicting the explicitly physical fact of their
equilibrium.  That is, the math cannot give two physical values.  In addition,
some Hamiltonians cause the calculations to blow up.  To give a simple
example, if the Hamiltonian $H_{0} \equiv \frac{p^2}{2 m} + \frac{m \omega^2
x^2}{2}$ for the (undamped) SHO is replaced with $H' \equiv - \frac{p^2}{2 m}
- \frac{m \omega^2 x^2}{2} = - H_{0}$, the eigenvalues become $- \hbar \omega
(n + \frac{1}{2})$.  These values cause the partition sum to diverge.  Since
$H_{0}$ and $H'$ are indistinguishably good time-evolution operators, we have
trouble reasoning {\em a priori} why only the former's eigenvalues should be 
used in the partition sum.

This leads us to consider use of the energy.  For normal (conservative system)
thermodynamics, there is no difference.  (For that matter, classical 
thermodynamics {\em always} speaks of energy, rather than of time evolution.)
In these systems, the energy can be seen as specifying one of the
time-evolution operators uniquely (i.e. the one that is identical with it).
Damped systems such as ours will usually have energy spectra different from
all Hamiltonian spectra.  Since the energy operator is still uniquely 
determined, we do not have the multiplicity problem that plagued us earlier.
Further, it makes sense physically that states should be weighted by something
with physical significance, i.e. the eigenvalues of energy, rather than 
something with none, i.e. the eigenvalues of time evolution.  (The 
time-evolution {\em operator} is of course tremendously relevant physically,
but its eigenvalues seem useless except when they are energy.)

\section{Results}

What are the results of this proposal?  The energy operator is unchanged by
the presence of a frictional force and remains ${\cal E} = 
\frac{p_{x}^{2}}{2 m} + \frac{k x^2}{2}$.  Since the CCR $[x, p_{x}] = i 
\hbar$ holds, we can (as done in \cite{ft:fest}) form the usual creation and 
destruction operators $a^{\dagger}, a$ with $[a, a^{\dagger}] = 1$.  This is 
sufficient to show that the eigenvalues of ${\cal E}$ are $\hbar \omega (n +
\frac{1}{2})$, as usual.

This means that, when we use the energy values in the partition sum, we get
exactly the statistical mechanics of the undamped oscillator.  This holds
whether the system is underdamped, critically damped, or overdamped.  Such
a solution certainly makes subsequent math simple; but does it make sense
physically?

\section{Discussion}

The use of the partition sum means that we are dealing with a canonical 
ensemble.  Although there are multiple ways to approach the canonical ensemble,
to make the identification $\beta = \frac{1}{k_{b} T}$ and thereby give the
math involved physical significance it is always necessary to model the system
as coupled to a heat bath.  In our case, we may hope that the sink into which
the oscillators' energy flows forms the bath, but it does not matter; if and
only if there is such a bath, the quantities calculated from the partition
function have physical and not merely mathematical meaning.  What this is
telling us, then, seems to be that the damping is countered by the presence
of the heat bath for the oscillators to absorb energy from (or emit it to).
They maintain the same equilibrium distribution as undamped oscillators would;
the ensemble is continuously losing energy to friction and gaining it back
from the bath.

In this context it is important to distinguish between the energy eigenvalues
and eigenstates being constant and being stationary.  Since $[{\cal E}, H] 
\neq 0$ for nonzero damping, the energy eigenstates will not be stationary.
Thus, a ket in one of these states at time $t$ will not have the same
energy eigenvalue at time $t + \Delta t$ (it is unlikely to be in an energy
eigenstate at all).  However, as ${\cal E}$ does not explicitly depend on time,
its eigenstates and eigenvalues are constant.  This means there is no
time-dependence difficulty in putting the eigenvalue spectrum into the
partition sum.  The oscillators (in isolation) make their way through the
eigenstates, which themselves are unchanged.  With the bath, apparently, the
oscillators are continuously restored at $t + dt$ to whatever combination of 
energy eigenstates they occupied at $t$.  The nonstationary nature of the
energy eigenstates would only be a problem if we considered a microcanonical
ensemble; in this case time-dependence at all stages of calculation seems
unavoidable.

\section{Confidence}

The above logic shows fairly convincingly that time-evolution operator 
eigenvalues are unfit candidates for the partition sum spectrum, and that
energy eigenvalues should be substituted.  It is certainly not a proof,
however.  Several objections have come up and been countered with varying 
degrees of surety.

First, it is {\em not} to be thought that the eigenvalues of the above 
${\cal E}$ can be used for an arbitrarily modified harmonic oscillator system.
As noted earlier, it is only because both the potential and the kinetic
energy of the damped system are identical with those of the regular system
that this can be done.  If one were to modify a harmonic oscillator with some
potential, the form of the energy would change and the new energy operator
would have to be used.  It is only elements such as friction that are dropped.

Second, it has been thought that the decaying harmonic oscillator might have 
the ability to experience canonical statistics even in the absence of an 
external heat bath due to the presence of both dissipative and regenerative 
terms.  That is, the energy lost in the $x$ dimension is gained in or made up 
for by that in the $y$ dimension.  Essentially, the heat bath is internalized.
One of the advantages of restricting the ensemble is that it seems to indicate
such an elegant equilibrium mechanism.  Since the $x$ and $y$ equations of
motion are the same up to the sign of $R$, whenever $x = y$ and $\dot{x} = 
\dot{y}$, $dE_{x} = - dE_{y}$.  If the energy gained each instant in the
$y$ motion is immediately transferred to the $x$ motion they can ``cancel''
each other's frictions and continue to do so forever.  The method of using
energy eigenvalues in the partition sum does not require that any
such thing is the case.

Nonetheless, it is quite possible that it implies that something of the sort
is occurring.  In fact, if we remove any external heat bath and assert that
the system still has canonical thermodynamics we are forcing such a 
relationship to be present, though only on a larger scale.  That is, if 
{\em all} the $x$-motions together lose $dE$ and {\em all} the 
$y$-motions together gain $dE$ then equilibrium can be maintained by multiple
transfers whether or not the individual $dE$s can be paired up.  Further, it
may be the case that the equilibrium distribution given by the partition sum
using energy eigenvalues {\em does} yield a situation in which $x$ and $y$
motions only occur in pairs, indicating that such a mechanism is quite likely.
Insofar as the elegance of the pairing makes its presence probable at all,
this is probable; and insofar as it does not, its lack is not a problem.

Thirdly, the use of the heat bath, whether internal or external, to feed
energy into the system is by no means exceptional.  Any normal thermodynamic
system reaches equilibrium via interactions, described by a small Hamiltonian
which is always omitted from the system's written Hamiltonian, among the
particles and between them and the walls.  During this process the bath will
either contribute energy to or take energy from the ensemble, generally in
large amounts.  Further, the bath continues to ``push'' the ensemble back
into equilibrium with energy additions and subtractions as the ensemble
continuously fluctuates.  The statement that the damped oscillator has a 
constant equilibrium distribution simply means that the bath is doing this
in a more regular fashion than usual, since the ensemble is always
``fluctuating'' (that is, decaying) in the same direction.  There is no reason
to doubt that this is a legitimate function of a heat bath.

In conclusion, while this solution is not rigorously proven, it satisfies
our criteria and there are no known overriding objections to it.

\newpage
\bibliography{main}
\bibliographystyle{plain}
  
