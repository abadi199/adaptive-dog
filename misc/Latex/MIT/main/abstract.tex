Several quantum mechanical phenomena can be described in terms of simple
harmonic oscillators modified by the inclusion of linear friction.  An
understanding of such oscillators is therefore of use.  The Hamiltonian
developed by Herman Feshbach \cite{ft:fest} to analyze these oscillators
has eigenvalues $\hbar \Omega (n_{A} - n_{B}) \pm \frac{i \hbar R}{2 m}
(n_{A} + n_{B} + 1)$, where $\Omega$ is the reduced frequency and $R$ measures
the strength of the friction.  Since the real part has no lower bound, the
partition sum immediately diverges.  The effects of the unprecedented
complexity of values in a partition function are unknown.  Several approaches
are examined, including separation and / or recoupling of the Hamiltonian,
placing restrictions on the ensemble membership, and studying the time
evolution and correlations of the system.
A solution is found after observing that the spectrum of the Hamiltonian cannotbe uniquely specified now that the Hamiltonian is no longer the energy, and so
use of Hamiltonian eigenvalues in the partition sum is improper.  By
determining that the energy eigenvalues are the proper input (leaving
traditional thermodynamics unchanged) we solve for the statistical mechanics
of the damped system, showing that the damping has no effect.
